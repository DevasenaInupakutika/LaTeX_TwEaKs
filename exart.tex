\documentclass[12pt]{article}                    % article class
 
\pagestyle{empty} 			   % no pagenumbers

\begin{document}                           % Begin document text
 
\section*{Introducing LaTeX}                % Print a section heading
LaTeX is not hard to learn.  With a few simple commands you
can produce a very nice looking document.
\textit{Here's how to put something in italics} and 
\textbf{here's how to make something boldface}.

{\small The text inside these braces is smaller than normal.} Now the size
is back to normal.  
Notice the effect of ``grouping'' (enclosing a section in braces). 
The change to a smaller font only applies inside the braces. 
\textit{This is a different use of braces than grouping. Here you are
using them to enclose the argument of a command!}
  
\subsection*{Dashes}                        % Print a subsection heading
Here's how to print dashes---the em-dash is written by typing three 
consecutive dashes; an en-dash (used between two numerals or in phone
numbers such as 764--9595) is written by typing two consecutive dashes. 

\subsection*{Quotes}                        % Print a subsection heading 
`Left quotes are written with the left quote (or grave) character (`); 
right quotes with the standard single quote (or prime) character (')'. 
``Double quotes are written by typing two consecutive quote characters 
of the desired flavor.'' For example, here's ``a quoted phrase.''
 
\end{document}                             % The required last line

\documentclass[12pt]{article}
\usepackage{amsmath}
\setlength{\parindent}{0pt} 
\setlength{\parskip}{10pt} % block paragraphs
\begin{document}

\section*{\centerline{Some Mathematical Expressions}}%unnumbered centered head

Two displayed equations. Note that using \$\$ gives the same result as 
using the displaymath environment.
 
$$x^2+y^2=z^2$$

\begin{displaymath} 
\sum_{i=1}^n a_i
\end{displaymath}
 
The expression $\int_a^b f(x)\, dx$ looks different in-line 
than it does displayed: $$\int_a^b f(x)\, dx$$ 

Note use of ``left'' and ``right'' commands in the equation below. They
make the following delimiter (in this case parentheses) grow to the
appropriate size.
\begin{equation}
\left( \frac{x+y}{z+2} \right)
\end{equation}

The middle line of this eqnarray is purposely not numbered: 
\begin{eqnarray}
x & = & 17y \\
y & > & a+b+c+d+ \nonumber \\
  &   & e+f+g
\end{eqnarray}

\begin{equation}
\binom{n}{k-1} + \binom{n}{k} = \binom{n+1}{k}
\end{equation}

\begin{equation}
\lim_{n\to\infty} {\sum_{k=1}^n 1/k^2} = \pi^2/6
\end{equation}

$4 \times 4$~matrix
\[ \left( \begin{array}{cccc}
1 & 0 & 0 & 0 \\
0 & 1 & 0 & 0 \\
0 & 0 & 1 & 0 \\
0 & 0 & 0 & 1 \end{array} \right)\]


\end{document}
